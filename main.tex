\documentclass[a4paper]{jpconf}
\usepackage{graphicx}
\begin{document}
\title{Design of a bend-twist coupled blade for a small wind turbine using multidisciplinary optimisation}

\author{Antariksh Dicholkar$^1$, Frederik Zahle$^1$ and Taeseong Kim$^2$}

\address{$^1$ DTU Wind Energy, DTU Ris{\o} campus, Frederiksborgvej 399, DK-4000 Roskilde}

\address{$^2$ DTU Wind Energy, Technical University of Denmark, Nils Koppels All{\'e}, Building 403, DK-2800 Kgs. Lyngby }

\ead{acdi@dtu.dk}
%%---------------------------------------------------------------------------------------------------------------------------
%%--------------------------------------------------------------------------------------------------------------------------
\begin{abstract}
The present study focuses on including bend-twist coupling (BTC) in the design of a 500W rotor by using a combination of parametric studies and a multidisciplinary optimisation (MDO) approach. The effectiveness of BTC is gauged through obtaining a significant decrease in the flapwise blade root bending moment accompanied by only a marginal decrease in the AEP, when compared with the baseline uncoupled turbine. Carbon outperforms glass for all fibre angles with regard to the amount of coupling seen in the blade cross-sections. Apart from flapwise bend-twist coupling, other secondary torsion couplings are also present. The steady state load response of the blade is assessed for varying positive fibre layup angles. An increase in the flapwise blade tip displacement and a reduction in the flapwise bending loads are observed. The HawtOpt2 aero-structural design tool is utilised to implement the MDO of the baseline blade.The spanwise fibre layup angle and laminate thickness distribution are the only structural design variables along with planform and operational variables to be provided with design freedom.The objective function seeks to minimise extreme flapwise bending loads and maximise the AEP through a weighted function. The performance of the optimised blades are compared for various combinations of the assigned weights.
\end{abstract}
%%--------------------------------------------------------------------------------------------------------------------------
%%-------------------------------------------------------------------------------------------------------------------------
\section{Introduction}
T

\section*{References}
\begin{thebibliography}{9}
\bibitem{iopartnum} IOP Publishing is to grateful Mark A Caprio, Center for Theoretical Physics, Yale University, for permission to include the {\tt iopart-num} \BibTeX package (version 2.0, December 21, 2006) with  this documentation. Updates and new releases of {\tt iopart-num} can be found on \verb"www.ctan.org" (CTAN). 
\end{thebibliography}

\end{document}


